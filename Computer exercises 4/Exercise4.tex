% Options for packages loaded elsewhere
\PassOptionsToPackage{unicode}{hyperref}
\PassOptionsToPackage{hyphens}{url}
%
\documentclass[
]{article}
\title{Prediction and Time Series Computer Exercise Week 4}
\author{Name: Nguyen Xuan Binh Student ID: 887799}
\date{11/25/2021}

\usepackage{amsmath,amssymb}
\usepackage{lmodern}
\usepackage{iftex}
\ifPDFTeX
  \usepackage[T1]{fontenc}
  \usepackage[utf8]{inputenc}
  \usepackage{textcomp} % provide euro and other symbols
\else % if luatex or xetex
  \usepackage{unicode-math}
  \defaultfontfeatures{Scale=MatchLowercase}
  \defaultfontfeatures[\rmfamily]{Ligatures=TeX,Scale=1}
\fi
% Use upquote if available, for straight quotes in verbatim environments
\IfFileExists{upquote.sty}{\usepackage{upquote}}{}
\IfFileExists{microtype.sty}{% use microtype if available
  \usepackage[]{microtype}
  \UseMicrotypeSet[protrusion]{basicmath} % disable protrusion for tt fonts
}{}
\makeatletter
\@ifundefined{KOMAClassName}{% if non-KOMA class
  \IfFileExists{parskip.sty}{%
    \usepackage{parskip}
  }{% else
    \setlength{\parindent}{0pt}
    \setlength{\parskip}{6pt plus 2pt minus 1pt}}
}{% if KOMA class
  \KOMAoptions{parskip=half}}
\makeatother
\usepackage{xcolor}
\IfFileExists{xurl.sty}{\usepackage{xurl}}{} % add URL line breaks if available
\IfFileExists{bookmark.sty}{\usepackage{bookmark}}{\usepackage{hyperref}}
\hypersetup{
  pdftitle={Prediction and Time Series Computer Exercise Week 4},
  pdfauthor={Name: Nguyen Xuan Binh Student ID: 887799},
  hidelinks,
  pdfcreator={LaTeX via pandoc}}
\urlstyle{same} % disable monospaced font for URLs
\usepackage[margin=1in]{geometry}
\usepackage{color}
\usepackage{fancyvrb}
\newcommand{\VerbBar}{|}
\newcommand{\VERB}{\Verb[commandchars=\\\{\}]}
\DefineVerbatimEnvironment{Highlighting}{Verbatim}{commandchars=\\\{\}}
% Add ',fontsize=\small' for more characters per line
\usepackage{framed}
\definecolor{shadecolor}{RGB}{248,248,248}
\newenvironment{Shaded}{\begin{snugshade}}{\end{snugshade}}
\newcommand{\AlertTok}[1]{\textcolor[rgb]{0.94,0.16,0.16}{#1}}
\newcommand{\AnnotationTok}[1]{\textcolor[rgb]{0.56,0.35,0.01}{\textbf{\textit{#1}}}}
\newcommand{\AttributeTok}[1]{\textcolor[rgb]{0.77,0.63,0.00}{#1}}
\newcommand{\BaseNTok}[1]{\textcolor[rgb]{0.00,0.00,0.81}{#1}}
\newcommand{\BuiltInTok}[1]{#1}
\newcommand{\CharTok}[1]{\textcolor[rgb]{0.31,0.60,0.02}{#1}}
\newcommand{\CommentTok}[1]{\textcolor[rgb]{0.56,0.35,0.01}{\textit{#1}}}
\newcommand{\CommentVarTok}[1]{\textcolor[rgb]{0.56,0.35,0.01}{\textbf{\textit{#1}}}}
\newcommand{\ConstantTok}[1]{\textcolor[rgb]{0.00,0.00,0.00}{#1}}
\newcommand{\ControlFlowTok}[1]{\textcolor[rgb]{0.13,0.29,0.53}{\textbf{#1}}}
\newcommand{\DataTypeTok}[1]{\textcolor[rgb]{0.13,0.29,0.53}{#1}}
\newcommand{\DecValTok}[1]{\textcolor[rgb]{0.00,0.00,0.81}{#1}}
\newcommand{\DocumentationTok}[1]{\textcolor[rgb]{0.56,0.35,0.01}{\textbf{\textit{#1}}}}
\newcommand{\ErrorTok}[1]{\textcolor[rgb]{0.64,0.00,0.00}{\textbf{#1}}}
\newcommand{\ExtensionTok}[1]{#1}
\newcommand{\FloatTok}[1]{\textcolor[rgb]{0.00,0.00,0.81}{#1}}
\newcommand{\FunctionTok}[1]{\textcolor[rgb]{0.00,0.00,0.00}{#1}}
\newcommand{\ImportTok}[1]{#1}
\newcommand{\InformationTok}[1]{\textcolor[rgb]{0.56,0.35,0.01}{\textbf{\textit{#1}}}}
\newcommand{\KeywordTok}[1]{\textcolor[rgb]{0.13,0.29,0.53}{\textbf{#1}}}
\newcommand{\NormalTok}[1]{#1}
\newcommand{\OperatorTok}[1]{\textcolor[rgb]{0.81,0.36,0.00}{\textbf{#1}}}
\newcommand{\OtherTok}[1]{\textcolor[rgb]{0.56,0.35,0.01}{#1}}
\newcommand{\PreprocessorTok}[1]{\textcolor[rgb]{0.56,0.35,0.01}{\textit{#1}}}
\newcommand{\RegionMarkerTok}[1]{#1}
\newcommand{\SpecialCharTok}[1]{\textcolor[rgb]{0.00,0.00,0.00}{#1}}
\newcommand{\SpecialStringTok}[1]{\textcolor[rgb]{0.31,0.60,0.02}{#1}}
\newcommand{\StringTok}[1]{\textcolor[rgb]{0.31,0.60,0.02}{#1}}
\newcommand{\VariableTok}[1]{\textcolor[rgb]{0.00,0.00,0.00}{#1}}
\newcommand{\VerbatimStringTok}[1]{\textcolor[rgb]{0.31,0.60,0.02}{#1}}
\newcommand{\WarningTok}[1]{\textcolor[rgb]{0.56,0.35,0.01}{\textbf{\textit{#1}}}}
\usepackage{graphicx}
\makeatletter
\def\maxwidth{\ifdim\Gin@nat@width>\linewidth\linewidth\else\Gin@nat@width\fi}
\def\maxheight{\ifdim\Gin@nat@height>\textheight\textheight\else\Gin@nat@height\fi}
\makeatother
% Scale images if necessary, so that they will not overflow the page
% margins by default, and it is still possible to overwrite the defaults
% using explicit options in \includegraphics[width, height, ...]{}
\setkeys{Gin}{width=\maxwidth,height=\maxheight,keepaspectratio}
% Set default figure placement to htbp
\makeatletter
\def\fps@figure{htbp}
\makeatother
\setlength{\emergencystretch}{3em} % prevent overfull lines
\providecommand{\tightlist}{%
  \setlength{\itemsep}{0pt}\setlength{\parskip}{0pt}}
\setcounter{secnumdepth}{-\maxdimen} % remove section numbering
\ifLuaTeX
  \usepackage{selnolig}  % disable illegal ligatures
\fi

\begin{document}
\maketitle

Exercise 4.3: A time series of carbon dioxide measurements from the
Mauna Loa volcano is given in the file MLCO2.txt. The length of the time
series is 216 months. Recall that, we studied this time series during
the third computer exercises. a) Using SARIMA processes, find the best
possible model to describe the time series MLCO2.

\begin{Shaded}
\begin{Highlighting}[]
\FunctionTok{library}\NormalTok{(forecast) }\CommentTok{\# Run every time you want to use functions of the forecast package}
\end{Highlighting}
\end{Shaded}

\begin{verbatim}
## Warning: package 'forecast' was built under R version 4.1.2
\end{verbatim}

\begin{verbatim}
## Registered S3 method overwritten by 'quantmod':
##   method            from
##   as.zoo.data.frame zoo
\end{verbatim}

\begin{Shaded}
\begin{Highlighting}[]
\NormalTok{MLCO2 }\OtherTok{\textless{}{-}} \FunctionTok{read.table}\NormalTok{(}\StringTok{"MLCO2.txt"}\NormalTok{,}\AttributeTok{header=}\NormalTok{T)}

\NormalTok{mlco2ts }\OtherTok{\textless{}{-}} \FunctionTok{ts}\NormalTok{(MLCO2}\SpecialCharTok{$}\NormalTok{MLCO2, }\AttributeTok{frequency =} \DecValTok{1}\NormalTok{)}


\FunctionTok{plot}\NormalTok{(mlco2ts, }\AttributeTok{xlab =} \StringTok{"Month"}\NormalTok{ ,}\AttributeTok{ylab =} \StringTok{"CO2 measurements"}\NormalTok{, }\AttributeTok{main =} \StringTok{"Time series of CO2 measurements from the Mauna Loa volcano"}\NormalTok{)}
\end{Highlighting}
\end{Shaded}

\includegraphics{Exercise4_files/figure-latex/unnamed-chunk-1-1.pdf}
According to the time series, the CO2 measurements display seasonality
and also a regular upward trend. We will now test which time series
model will fit best by studying the time series' ACF and PACF

\begin{Shaded}
\begin{Highlighting}[]
\FunctionTok{par}\NormalTok{(}\AttributeTok{mfrow=}\FunctionTok{c}\NormalTok{(}\DecValTok{1}\NormalTok{,}\DecValTok{2}\NormalTok{),}\AttributeTok{mar=}\FunctionTok{c}\NormalTok{(}\FloatTok{2.5}\NormalTok{,}\FloatTok{2.5}\NormalTok{,}\FloatTok{3.5}\NormalTok{,}\FloatTok{1.5}\NormalTok{))}
\FunctionTok{acf}\NormalTok{(mlco2ts,}\AttributeTok{main=}\StringTok{"ACF"}\NormalTok{, }\AttributeTok{lag.max=}\DecValTok{50}\NormalTok{) }
\FunctionTok{pacf}\NormalTok{(mlco2ts,}\AttributeTok{main=}\StringTok{"PACF"}\NormalTok{, }\AttributeTok{lag.max=}\DecValTok{50}\NormalTok{) }
\end{Highlighting}
\end{Shaded}

\includegraphics{Exercise4_files/figure-latex/unnamed-chunk-2-1.pdf}
Analysis: The time series is seasonal ACF decays exponentially (although
it may appear as linear decay at first 50 lags) along the seasonal lags
and PACF has cut off at lag 2. The sample PACF with lags 4, 5, 8, 9, 12,
13 reach over the blue lines indicating statistical significance
=\textgreater{} AR(2) should be a component of SARIMA model. Now we use
auto sarima model on the time series.

\begin{Shaded}
\begin{Highlighting}[]
\NormalTok{model\_mlco2 }\OtherTok{\textless{}{-}} \FunctionTok{auto.arima}\NormalTok{(mlco2ts)}
\NormalTok{model\_mlco2}
\end{Highlighting}
\end{Shaded}

\begin{verbatim}
## Series: mlco2ts 
## ARIMA(2,1,1) with drift 
## 
## Coefficients:
##          ar1      ar2      ma1   drift
##       1.5336  -0.8358  -0.9357  0.0778
## s.e.  0.0368   0.0364   0.0191  0.0093
## 
## sigma^2 estimated as 0.3561:  log likelihood=-193.69
## AIC=397.38   AICc=397.66   BIC=414.23
\end{verbatim}

From the auto.arima function, it needs first order differencing of
non-seasonal component in order to remove the linearly upward trend of
the time series. Also, AR(2) is present in the prediction
=\textgreater{} SARIMA(2, 1, 1)(P, H, Q){[}12{]} is a candidate. We will
try the auto.arima function for the time series of frequency = 12

\begin{Shaded}
\begin{Highlighting}[]
\NormalTok{mlco2SeasonTs }\OtherTok{\textless{}{-}} \FunctionTok{ts}\NormalTok{(MLCO2}\SpecialCharTok{$}\NormalTok{MLCO2, }\AttributeTok{frequency =} \DecValTok{12}\NormalTok{)}
\NormalTok{model\_mlco2Season }\OtherTok{\textless{}{-}} \FunctionTok{auto.arima}\NormalTok{(mlco2SeasonTs)}
\NormalTok{model\_mlco2Season}
\end{Highlighting}
\end{Shaded}

\begin{verbatim}
## Series: mlco2SeasonTs 
## ARIMA(0,1,2)(2,1,0)[12] 
## 
## Coefficients:
##           ma1      ma2     sar1     sar2
##       -0.3609  -0.1808  -0.6505  -0.3520
## s.e.   0.0679   0.0672   0.0679   0.0705
## 
## sigma^2 estimated as 0.1299:  log likelihood=-82.17
## AIC=174.34   AICc=174.65   BIC=190.91
\end{verbatim}

Since the model has been differenced with AR(2), seasonal differencing
will depend on moving average, which is SMA(2){[}12{]} The candidate
model now is SARIMA(2, 1, 1)(0, 1, 2){[}12{]}. We need to verify if it
is correct model

\begin{Shaded}
\begin{Highlighting}[]
\NormalTok{modelSARIMA }\OtherTok{\textless{}{-}} \FunctionTok{Arima}\NormalTok{(mlco2ts,}\AttributeTok{order=}\FunctionTok{c}\NormalTok{(}\DecValTok{2}\NormalTok{,}\DecValTok{1}\NormalTok{,}\DecValTok{1}\NormalTok{), }\AttributeTok{seasonal=}\FunctionTok{list}\NormalTok{(}\AttributeTok{order=}\FunctionTok{c}\NormalTok{(}\DecValTok{0}\NormalTok{,}\DecValTok{1}\NormalTok{,}\DecValTok{2}\NormalTok{),}\AttributeTok{periodc=}\DecValTok{12}\NormalTok{))}
\FunctionTok{par}\NormalTok{(}\AttributeTok{mfrow=}\FunctionTok{c}\NormalTok{(}\DecValTok{1}\NormalTok{,}\DecValTok{2}\NormalTok{),}\AttributeTok{mar=}\FunctionTok{c}\NormalTok{(}\FloatTok{2.5}\NormalTok{,}\FloatTok{2.5}\NormalTok{,}\FloatTok{3.5}\NormalTok{,}\FloatTok{1.5}\NormalTok{))}
\FunctionTok{acf}\NormalTok{(modelSARIMA}\SpecialCharTok{$}\NormalTok{res,}\AttributeTok{main=}\StringTok{"ACF of residuals"}\NormalTok{,}\AttributeTok{lag.max=}\DecValTok{50}\NormalTok{)}
\FunctionTok{pacf}\NormalTok{(modelSARIMA}\SpecialCharTok{$}\NormalTok{res, }\AttributeTok{main=}\StringTok{"PACF of residuals"}\NormalTok{, }\AttributeTok{lag.max=}\DecValTok{50}\NormalTok{)}
\end{Highlighting}
\end{Shaded}

\includegraphics{Exercise4_files/figure-latex/unnamed-chunk-5-1.pdf}
There are about 3 significant value in residual ACF and 3 significant
values in residual PACF, which is few compared to the large number of
sample data. It could be a sign that this SARIMA model is correct Now we
utilize Ljung Box test Number of fitted parameters: k = 2+1+2 = 5

\begin{Shaded}
\begin{Highlighting}[]
\NormalTok{k }\OtherTok{\textless{}{-}} \DecValTok{5}
\NormalTok{mlco2\_bl }\OtherTok{\textless{}{-}} \FunctionTok{rep}\NormalTok{(}\ConstantTok{NA}\NormalTok{,}\DecValTok{47}\NormalTok{)}
\ControlFlowTok{for}\NormalTok{ (i }\ControlFlowTok{in} \DecValTok{1}\SpecialCharTok{:}\DecValTok{47}\NormalTok{) \{}
\NormalTok{  mlco2\_bl[i]}\OtherTok{=}\FunctionTok{Box.test}\NormalTok{(modelSARIMA}\SpecialCharTok{$}\NormalTok{res,}\AttributeTok{lag=}\NormalTok{(i}\SpecialCharTok{+}\NormalTok{k),}\AttributeTok{fitdf=}\NormalTok{k, }\AttributeTok{type=}\StringTok{"Ljung{-}Box"}\NormalTok{)}\SpecialCharTok{$}\NormalTok{p.value}
\NormalTok{\}}
\FunctionTok{round}\NormalTok{(mlco2\_bl, }\DecValTok{3}\NormalTok{)}
\end{Highlighting}
\end{Shaded}

\begin{verbatim}
##  [1] 0.148 0.351 0.517 0.156 0.191 0.158 0.217 0.258 0.325 0.411 0.500 0.439
## [13] 0.504 0.464 0.359 0.382 0.448 0.263 0.318 0.325 0.381 0.376 0.399 0.456
## [25] 0.486 0.540 0.577 0.629 0.595 0.554 0.595 0.642 0.676 0.625 0.649 0.608
## [37] 0.529 0.566 0.552 0.597 0.640 0.615 0.594 0.624 0.647 0.662 0.684
\end{verbatim}

\begin{Shaded}
\begin{Highlighting}[]
\FunctionTok{which}\NormalTok{(mlco2\_bl }\SpecialCharTok{\textless{}} \FloatTok{0.05}\NormalTok{)}
\end{Highlighting}
\end{Shaded}

\begin{verbatim}
## integer(0)
\end{verbatim}

There are no p-values of the SARIMA model residuals that is smaller than
0.05 in the Ljung-Box test =\textgreater{} The null hypothesis of
Ljung-Box is rejected. The SARIMA(2, 1, 1)(0, 1, 2){[}12{]} will be
chosen as the model for future prediction. Now we compare original time
series and the SARIMA fit model

\begin{Shaded}
\begin{Highlighting}[]
\NormalTok{fit.mlco2 }\OtherTok{\textless{}{-}} \FunctionTok{fitted}\NormalTok{(modelSARIMA)}

\FunctionTok{plot}\NormalTok{(fit.mlco2,}\AttributeTok{type=}\StringTok{"b"}\NormalTok{,}\AttributeTok{col=}\StringTok{"blue"}\NormalTok{, }\AttributeTok{ylab=}\StringTok{"CO2 measurements"}\NormalTok{,}\AttributeTok{xlab=}\StringTok{"Month"}\NormalTok{)}

\FunctionTok{lines}\NormalTok{(mlco2ts,}\AttributeTok{col=}\StringTok{"red"}\NormalTok{,}\AttributeTok{type=}\StringTok{"b"}\NormalTok{)}
\FunctionTok{legend}\NormalTok{(}\DecValTok{150}\NormalTok{,}\DecValTok{20}\NormalTok{, }\AttributeTok{legend=}\FunctionTok{c}\NormalTok{(}\StringTok{"Original time series"}\NormalTok{, }\StringTok{"SARIMA fit"}\NormalTok{),}
\AttributeTok{col=}\FunctionTok{c}\NormalTok{(}\StringTok{"red"}\NormalTok{,}\StringTok{"blue"}\NormalTok{),}\AttributeTok{lty=}\FunctionTok{c}\NormalTok{(}\DecValTok{1}\NormalTok{,}\DecValTok{1}\NormalTok{),}\AttributeTok{cex=}\FloatTok{0.8}\NormalTok{)}
\end{Highlighting}
\end{Shaded}

\includegraphics{Exercise4_files/figure-latex/unnamed-chunk-7-1.pdf} b)
Make 2 and 24 time step predictions by using the model chosen in a).
Study the goodness of the predictions 2 time step predictions by using
the model in a)

\begin{Shaded}
\begin{Highlighting}[]
\NormalTok{currentSARIMA }\OtherTok{=} \FunctionTok{arima}\NormalTok{(mlco2ts[}\DecValTok{1}\SpecialCharTok{:}\DecValTok{216}\NormalTok{],}\AttributeTok{order=}\FunctionTok{c}\NormalTok{(}\DecValTok{2}\NormalTok{,}\DecValTok{1}\NormalTok{,}\DecValTok{1}\NormalTok{), }\AttributeTok{seasonal=}\FunctionTok{list}\NormalTok{(}\AttributeTok{order=}\FunctionTok{c}\NormalTok{(}\DecValTok{0}\NormalTok{,}\DecValTok{1}\NormalTok{,}\DecValTok{2}\NormalTok{),}\AttributeTok{period=}\DecValTok{12}\NormalTok{) )}
\NormalTok{prediction2steps }\OtherTok{\textless{}{-}}\FunctionTok{forecast}\NormalTok{(currentSARIMA,}\AttributeTok{h=}\DecValTok{2}\NormalTok{,}\AttributeTok{level=}\ConstantTok{FALSE}\NormalTok{)}\SpecialCharTok{$}\NormalTok{mean}

\FunctionTok{plot}\NormalTok{(mlco2ts,}\AttributeTok{col=}\StringTok{"red"}\NormalTok{,}\AttributeTok{type=}\StringTok{"b"}\NormalTok{,}
     \AttributeTok{ylab=}\StringTok{"CO2 measurements"}\NormalTok{,}\AttributeTok{xlab=}\StringTok{"Months"}\NormalTok{,}\AttributeTok{main=}\StringTok{"CO2 measure from Mauna Loa volcano: 2 step prediction"}\NormalTok{)}
\FunctionTok{lines}\NormalTok{(prediction2steps,}\AttributeTok{col=}\StringTok{"blue"}\NormalTok{,}\AttributeTok{type=}\StringTok{"b"}\NormalTok{)}
\end{Highlighting}
\end{Shaded}

\includegraphics{Exercise4_files/figure-latex/unnamed-chunk-8-1.pdf}

24 time step predictions by using the model in a)

\begin{Shaded}
\begin{Highlighting}[]
\NormalTok{currentSARIMA }\OtherTok{=} \FunctionTok{arima}\NormalTok{(mlco2ts[}\DecValTok{1}\SpecialCharTok{:}\DecValTok{216}\NormalTok{],}\AttributeTok{order=}\FunctionTok{c}\NormalTok{(}\DecValTok{2}\NormalTok{,}\DecValTok{1}\NormalTok{,}\DecValTok{1}\NormalTok{), }\AttributeTok{seasonal=}\FunctionTok{list}\NormalTok{(}\AttributeTok{order=}\FunctionTok{c}\NormalTok{(}\DecValTok{0}\NormalTok{,}\DecValTok{1}\NormalTok{,}\DecValTok{2}\NormalTok{),}\AttributeTok{period=}\DecValTok{12}\NormalTok{) )}
\NormalTok{prediction2steps }\OtherTok{\textless{}{-}}\FunctionTok{forecast}\NormalTok{(currentSARIMA,}\AttributeTok{h=}\DecValTok{24}\NormalTok{,}\AttributeTok{level=}\ConstantTok{FALSE}\NormalTok{)}\SpecialCharTok{$}\NormalTok{mean}

\FunctionTok{plot}\NormalTok{(mlco2ts,}\AttributeTok{col=}\StringTok{"red"}\NormalTok{,}\AttributeTok{type=}\StringTok{"b"}\NormalTok{, }\AttributeTok{xlim=}\FunctionTok{c}\NormalTok{(}\DecValTok{0}\NormalTok{, }\DecValTok{250}\NormalTok{), }\AttributeTok{ylim =}\FunctionTok{c}\NormalTok{(}\DecValTok{10}\NormalTok{,}\DecValTok{40}\NormalTok{),}
     \AttributeTok{ylab=}\StringTok{"CO2 measurements"}\NormalTok{,}\AttributeTok{xlab=}\StringTok{"Months"}\NormalTok{,}\AttributeTok{main=}\StringTok{"CO2 measure from Mauna Loa volcano: 24 step prediction"}\NormalTok{)}
\FunctionTok{lines}\NormalTok{(prediction2steps,}\AttributeTok{col=}\StringTok{"blue"}\NormalTok{,}\AttributeTok{type=}\StringTok{"b"}\NormalTok{)}
\end{Highlighting}
\end{Shaded}

\includegraphics{Exercise4_files/figure-latex/unnamed-chunk-9-1.pdf}

\begin{Shaded}
\begin{Highlighting}[]
\NormalTok{currentSARIMA }\OtherTok{=} \FunctionTok{arima}\NormalTok{(mlco2ts[}\DecValTok{1}\SpecialCharTok{:}\DecValTok{216}\NormalTok{],}\AttributeTok{order=}\FunctionTok{c}\NormalTok{(}\DecValTok{2}\NormalTok{,}\DecValTok{1}\NormalTok{,}\DecValTok{1}\NormalTok{), }\AttributeTok{seasonal=}\FunctionTok{list}\NormalTok{(}\AttributeTok{order=}\FunctionTok{c}\NormalTok{(}\DecValTok{0}\NormalTok{,}\DecValTok{1}\NormalTok{,}\DecValTok{2}\NormalTok{),}\AttributeTok{period=}\DecValTok{12}\NormalTok{) )}
\NormalTok{prediction2steps }\OtherTok{\textless{}{-}}\FunctionTok{forecast}\NormalTok{(currentSARIMA,}\AttributeTok{h=}\DecValTok{100}\NormalTok{,}\AttributeTok{level=}\ConstantTok{FALSE}\NormalTok{)}\SpecialCharTok{$}\NormalTok{mean}

\FunctionTok{plot}\NormalTok{(mlco2ts,}\AttributeTok{col=}\StringTok{"red"}\NormalTok{,}\AttributeTok{type=}\StringTok{"b"}\NormalTok{, }\AttributeTok{xlim=}\FunctionTok{c}\NormalTok{(}\DecValTok{0}\NormalTok{, }\DecValTok{320}\NormalTok{), }\AttributeTok{ylim =}\FunctionTok{c}\NormalTok{(}\DecValTok{10}\NormalTok{,}\DecValTok{45}\NormalTok{),}
     \AttributeTok{ylab=}\StringTok{"CO2 measurements"}\NormalTok{,}\AttributeTok{xlab=}\StringTok{"Months"}\NormalTok{,}\AttributeTok{main=}\StringTok{"CO2 measure from Mauna Loa volcano: 100 step prediction"}\NormalTok{)}
\FunctionTok{lines}\NormalTok{(prediction2steps,}\AttributeTok{col=}\StringTok{"blue"}\NormalTok{,}\AttributeTok{type=}\StringTok{"b"}\NormalTok{)}
\end{Highlighting}
\end{Shaded}

\includegraphics{Exercise4_files/figure-latex/unnamed-chunk-10-1.pdf} We
can see that prediction of SARIMA(2,1,1)(0,1,2){[}12{]} is quite good in
predicting the future of amount of CO2, since this model has passed the
Ljung-Box test. Therefore, I believe the prediction for time steps of 2
and 24 are likely to be correct under this SARIMA model

\end{document}
